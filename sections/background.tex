\newpage
\section{Background - Basic concepts}
	\subsection{The Simplex Method}

\qquad The \textbf{Simplex Method}, developed by George Dantzig, incorporates both \textit{optimality} and \textit{feasibility} tests to find the optimal solution(s) to a linear program (if one exists). 

Geometrically, the Simplex Method
proceeds from an initial extreme point to an adjacent extreme point until no adjacent extreme
point is more optimal. \\

\qquad To implement the Simplex Method we first separate the \textit{decision} and \textit{slack} variables into two non-overlapping sets that we call the \textbf{independent} and \textbf{dependent} sets. For the particular linear programs we consider, the original independent set will consist of the decision variables, and the slack variables will belong to the dependent set. \\[6pt]

\textbf{The Algorithm:}
\begin{enumerate}
	\item Tableau Format: Place the linear program in Tableau Format, as explained later.
	\item Initial Extreme Point: The Simplex Method begins with a known extreme point, usually
	the origin (0, 0).
	\item Optimality Test: Determine whether an adjacent intersection point improves the value of the objective function. If not, the current extreme point is optimal. If an improvement is possible, the optimality test determines which variable currently in the independent set (having value zero) should \textit{enter} the dependent set and become nonzero.
	\item Feasibility Test: To find a new intersection point, one of the variables in the dependent set must \textit{exit} to allow the entering variable from Step 3 to become dependent. The feasibility test determines which current dependent variable to choose for exiting, ensuring feasibility.
	\item Pivot: Form a new, equivalent system of equations by eliminating the new dependent variable from the equations do not contain the variable that exited in Step 4. Then set
	the new independent variables to zero in the new system to find the values of the new dependent variables, thereby determining an intersection point.
	\item Repeat Steps 3 - 5 until the extreme point is optimal.
	
\end{enumerate}

	\subsection{Place holder}
