\newpage
\section{PROBLEM I. [Industry - Manufacutring]}
	\subsection{Summary of the Problem}

	\qquad This problem revolves around minimizing the production output of an industrial firm F. Specifically, the firm is tasked with manufacturing \textcolor{cyan}{\textit{n}} distinct products. Each product necessitates varying quantities of components, sourced from \textcolor{cyan}{\textit{m}} external suppliers, each carrying a unit cost of \textcolor{cyan}{\textit{b}}. Before the product demands are determined, the firm must make decisions regarding the quantity of components to procure. Following this procurement phase, the production commences and the product demands are disclosed. Corresponding to each product, there exists a selling price, denoted as \textcolor{cyan}{\textit{q}}. In the event of an excess in component procurement beyond the production requirements, salvage components must be resold at a price of \textcolor{cyan}{\textit{s}} to mitigate costs. Additionally, any unmet product demands incur an extra cost of \textcolor{cyan}{\textit{l}} per unit to fulfill.

	\qquad This problem constitutes a 2-stage stochastic linear programming (2-SLP) challenge, where the product demands introduce uncertainty. The primary aim across the two stages involves minimizing the expenses linked to both pre-ordering components and actual production. However, as stipulated in the assignment, this problem necessitates recourse action, transforming it into a 2-stage stochastic linear programming problem with recourse (2-SLPWR) with the request of finding the optimal value of $x,y \in \mathbb{R}^m$ and $z \in \mathbb{R}^n$. The comprehensive formulation process and subsequent solution will be detailed further below.
	
	\subsection{Model formulation}
		\subsubsection{Notations}
		
		\qquad For ease of reading, the tables below summarize the notation of parameters and decision variables used in 
the formulation of this report.

		% Parameter table
		\begin{center}
		\begin{tabular}{|c|l|}
		\hline
		\textbf{Symbol} & \textbf{Definition}\\
		\hline 
		\multirow{1}{*}{$n$} & \multirow{1}{*}{Number of products}\\
		\hline
		\multirow{1}{*}{$m$} & \multirow{1}{*}{Number of parts}\\
		\hline
		\multirow{1}{*}{$i, j$} & \multirow{1}{*}{Index of products and parts, respectively}\\
		\hline
		\multirow{1}{*}{$b$} & \multirow{1}{*}{The set of preorder cost of parts}\\
		\hline
		\multirow{1}{*}{$q$} & \multirow{1}{*}{Selling price per unit of product $i$}\\
		\hline
		\multirow{1}{*}{$l$} & \multirow{1}{*}{The set of additional cost to satisfy a demand per unit}\\
		\hline
		\multirow{1}{*}{$A$} & \multirow{1}{*}{The matrix representing parts needed for each product}\\
		\hline
		\multirow{1}{*}{$a_{ij}$} & \multirow{1}{*}{Number of part $j$ needed for product $i$}\\
		\hline
		\multirow{1}{*}{$s$} & \multirow{1}{*}{The set of salvage part selling price}\\
		\hline
		\multirow{1}{*}{$D$} & \multirow{1}{*}{The set of the demand of the product}\\
		\hline
		\multirow{1}{*}{$S$} & \multirow{1}{*}{Number of scenarios}\\
		\hline
		\multirow{1}{*}{$p_s$} & \multirow{1}{*}{Probability of scenario $s$}\\
		\hline
		\end{tabular}
		\end{center}


		% Decision variable table
		\begin{center}
		\begin{tabular}{|c|l|}
			\hline
			\textbf{Symbol} & \textbf{Definition}\\
			\hline 
			\multirow{1}{*}{$x$} & \multirow{1}{*}{The set of preordering quantities}\\
			\hline
			\multirow{1}{*}{$z$} & \multirow{1}{*}{The set of manufactured products}\\
			\hline
			\multirow{1}{*}{$y$} & \multirow{1}{*}{The set of salvage parts}\\
			\hline
		\end{tabular}
		\end{center}

		\subsubsection{First Stage}

		\qquad The initial stage primarily focuses on ensuring the preordered quantity of parts is sufficient to accommodate future, yet uncertain, demands while avoiding surplus or deficit, which could adversely impact the firm's profitability. The decision variable, denoted as $x$, constitutes a \textit{here-and-now} determination that must be made prior to the revelation of actual demand. The preordering cost is represented as a linear function:
		
		$$f(x) = b^T x$$
		
		However, as the problem at hand pertains to a 2-SLPWR (2-stage stochastic linear programming problem with recourse), probability functions linked to distinct scenarios are incorporated to fine-tune the original constraints. Consequently, the quantities denoted by $x$ can be determined by solving the following optimization problem:

		\begin{align*}
			\text{minimize } & g(x, y, z) = b^T x + Q(x) = b^T x + E[Z(z)]
		\end{align*}

		Here, $Q(x) = E_{\omega}[Z] = \sum_{i=1}^{n} p_i c_i z_i$ is taken with respect to the probability distribution of $\omega = D$. $Q(x)$ represents the expected value of the optimal production cost (the optimal solution to the second-stage problem detailed below). Therefore, $Q(x) = \sum_{k=1}^{S} p_k (c^T z_k - s^T y_k)$, where $z_{k}$ and $y_{k}$ denote the set of manufactured products and the quantity of salvage parts in scenario $k$, respectively.

		\subsubsection{Second Stage}

		\qquad Subsequent to the initial stage, the actual demand $d=(d_1,d_2,...,d_n)$ for the products is realized. Consequently, the optimal production plan is derived by solving the stochastic linear program involving decision variables $x$ and $y$:

		\begin{align*}
			\text{minimize } & Z = \sum^n_{i=1} (l_i - q_i) z_i - \sum^m_{j=1} s_j y_j \\
			\text{subject to } & y_j = x_j - \sum_{i=1}^n a_{ij} z_i, \quad j=1,...,m \\
			& 0 \leq z_i \leq d_i, \quad i=1,...,n \\
			& y_j \geq 0, \quad j=1,...,m
		\end{align*}

		\subsubsection{Grand model}

		\qquad It can be observed that to solve the first stage problem, the optimal solution of the second stage is required. However, the optimal solutoin of the second stage again, needs the value of $x$ of the first stage. Here we counter the interdependence between the two stages. Therefore, to solve this problem, it is essential to combine the two stages into one \textit{grand objective function} and let the unknown demand follows the binomial distribution specificied in the description of the assignment. Then, the quantities $x$ can be decided and use that to solve the second stage again once the actual demand is realized. The grand objective function is described below:

		\begin{align*}
			\text{min } & g(x,y,z) = b^Tx + \sum_{k=1}^{S} p_k (c^T z_k - s^T y_k)\\
			\text{s.t. } & y_{k}=x- A^T z_k, \quad k=1,...,S \\
			& x \geq 0 \\
			& 0 \leq z_{k} \leq d_{k}, \quad k=1,...,S \\
			& y_k \geq 0, \quad k=1,...,S
		\end{align*}

		\subsubsection{Solution}

		\qquad In the description of the assignment, specific value of the parameters are given as follows:

		\begin{itemize}
			\item n = 8
			\item m = 5
			\item S = 2
			\item $p_s$ = 1/2
			\item Values of b, l, q, s and A are randomized 
			\item Random demand vector $D$ and its density $p_i$ follows the binomial distribution \textbf{Bin(10,$\frac{1}{2}$)}
		\end{itemize}

		\qquad To solve this 2-SLPWR problem, our group decided to employ Gurobi - a powerful mathematical optimization solver in Python programming language with Gurobi API. The source code has been included in our assignment submission. 

		\qquad For the values of b, l, q, s and A, a function to randomize integer is used to generate the set of said values. To be specific, the ranges for each one is listed below:

		\begin{itemize}
			\item Preordering cost $b$: 50 -> 100
			\item Additional producing cost $l$: 100 -> 200
			\item Selling price per unit $q$: 1000 -> 2000
			\item Salvage part selling price $s$: 30 -> price of the corresponding part (for part $j$ ranges from 1 to $m$, $s_j$ < $b_j$ as specificied in the assignment)
			\item Number of parts required for each product $A$: the number of each part required for a product will ranges from 0 -> 5.
		\end{itemize}

		\qquad Then, a model for the first stage are defined identically to the grand model in section 4.2.4 above. The model is then optimized with demand vector $D$ randomized following the binomial distribution to find quantities $x$. After that, a second model for the second stage is defined according to section 4.2.3. With that, a vector of $d$ is hard-coded to represent the fact that the actual demand has been revealed, $y$ and $z$ then can be found by solving the optimizing problem.

		\qquad For example, an instance of our model having the following parameters:

		\begin{itemize}
			\item $b$ = [74 50 56 66 51]
			\item $l$ = [138 131 196 188 169 127 164 101]
			\item $q$ = [1399 1122 1639 1410 1420 1572 1261 1377]
			\item $s$ = [32 32 41 42 32]
			\item $A = 	\begin{bmatrix}
							1 & 4 & 0 & 2 & 4 \\
							3 & 1 & 3 & 2 & 0 \\
							3 & 0 & 4 & 2 & 4 \\
							3 & 1 & 2 & 1 & 4 \\
							3 & 2 & 3 & 1 & 1 \\
							2 & 0 & 3 & 0 & 2 \\
							0 & 4 & 0 & 3 & 1 \\
							3 & 0 & 2 & 2 & 1 \\
						\end{bmatrix}$
			\item $D = 	\begin{bmatrix}
							5 & 4 & 5 & 5 & 4 & 6 & 4 & 6 \\
							6 & 2 & 6 & 6 & 4 & 4 & 6 & 5 \\
						\end{bmatrix}$
		\end{itemize}

		\qquad With those randomzed value, the optimal vector $x$ is found: [89 64 84 66 95]. Then second stage is solved with the known demand, which is set to [6 9 8 8 4 4 6 5] in our code, getting the result of $z$ and $y$ as follows: [5 3 5 8 4 4 6 5], [1 1 5 0 0].
